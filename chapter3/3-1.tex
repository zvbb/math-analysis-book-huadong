\documentclass{article}
\usepackage{mathtools} 
\usepackage{fontspec}
\usepackage[UTF8]{ctex}
\usepackage{amsthm}
\usepackage{mdframed}
\usepackage{xcolor}
\usepackage{amssymb}
\usepackage{amsmath}


% 定义新的带灰色背景的说明环境 zremark
\newmdtheoremenv[
  backgroundcolor=gray!10,
  % 边框与背景一致,边框线会消失
  linecolor=gray!10
]{zremark}{说明}

% 通用矩阵命令: \flexmatrix{矩阵名}{元素符号}{行数}{列数}
\newcommand{\flexmatrix}[4]{
  \[
  #1 = \begin{pmatrix}
    #2_{11}     & #2_{12}     & \cdots & #2_{1#4}   \\
    #2_{21}     & #2_{22}     & \cdots & #2_{2#4}   \\
    \vdots      & \vdots      & \ddots & \vdots     \\
    #2_{#31}    & #2_{#32}    & \cdots & #2_{#3#4}
  \end{pmatrix}
  \]
}

% 简化版命令(默认矩阵名为A,元素符号为a): \quickmatrix{行数}{列数}
\newcommand{\quickmatrix}[2]{\flexmatrix{A}{a}{#1}{#2}}

\begin{document}
\title{习题 3.1}
\author{张志聪}
\maketitle

\section*{1}

\begin{itemize}
  \item (1) $\lim\limits_{x \to +\infty} \frac{6x + 5}{x} = 6$;

        不放限制$x > 0$,对任意$\epsilon > 0$,取$M = \frac{5}{\epsilon}$,
        则当$x > M$时,有
        \begin{align*}
          |\frac{6x + 5}{x} - 6| & = \frac{5}{x} < \epsilon
        \end{align*}
        所以$\lim\limits_{x \to +\infty} \frac{6x + 5}{x} = 6$。

  \item (2) $\lim\limits_{x \to 2} (x^2 - 6x + 10) = 2$;

        我们有
        \begin{align*}
          |(x^2 - 6x + 10) - 2|
           & = |x^2 - 6x + 8|   \\
           & = |(x - 2)(x - 4)|
        \end{align*}

        若限制$x$于$|x - 2| < 1$,则
        \begin{align*}
          |x - 4| \leq |x - 2| + 2 < 1 + 2 = 3
        \end{align*}
        $\therefore$
        \begin{align*}
          |(x^2 - 6x + 10) - 2| & = |(x - 2)(x - 4)| \\
                                & < 3 |x - 2|
        \end{align*}
        对任意的$\epsilon$,令
        \begin{align*}
          3 |x - 2| < \epsilon \\
          |x - 2| < \frac{\epsilon}{3}
        \end{align*}
        于是,只要取$\delta = min(1, \frac{\epsilon}{3})$,则当$0 < |x - 2| < \delta$时,
        就有
        \begin{align*}
          |(x^2 - 6x + 10) - 2| < \epsilon
        \end{align*}
        所以$\lim\limits_{x \to 2} (x^2 - 6x + 10) = 2$。

  \item $\lim\limits_{x \to \infty} \frac{x^2 - 5}{x^2 - 1} = 1$;

        不妨限制$|x| > 1$。

        我们有
        \begin{align*}
          |\frac{x^2 - 5}{x^2 -1} - 1|
           & = |\frac{-4}{x^2 - 1}| \\
           & = \frac{4}{x^2 - 1}
        \end{align*}

        对任意的$\epsilon > 0$,令
        \begin{align*}
          \frac{4}{x^2 - 1} < \epsilon \\
          4 < (x^2 - 1) \epsilon       \\
          \frac{4}{\epsilon} + 1< x^2  \\
          \sqrt{\frac{4}{\epsilon} + 1} < |x|
        \end{align*}
        于是,只要取$M = max(1, \sqrt{\frac{4}{\epsilon} + 1})$,
        则当$|x| > M$时,有
        \begin{align*}
          |\frac{x^2 - 5}{x^2 -1} - 1| & = \frac{4}{x^2 - 1} < \epsilon
        \end{align*}
        所以$\lim\limits_{x \to \infty} \frac{x^2 - 5}{x^2 - 1} = 1$。

  \item $\lim\limits_{x \to 2^{-}} \sqrt{4 - x^2} = 0$;

        不妨限制$0 < x < 2$。

        我们有
        \begin{align*}
          |\sqrt{4 - x^2} - 0|
           & = \sqrt{4 - x^2}                  \\
           & = \sqrt{2 + x} \cdot \sqrt{2 - x} \\
           & < 2 \sqrt{2 - x}
        \end{align*}
        对任意$\epsilon > 0$,令
        \begin{align*}
          2 \sqrt{2 - x} < \epsilon \\
          4 (2 - x) < \epsilon^2    \\
          (2 - x) < \frac{\epsilon^2}{4}
        \end{align*}
        于是取$\delta = \frac{\epsilon^2}{4}$,
        则当$0 < 2 - x < \delta$即$2 - \delta < x < 2$时,
        \begin{align*}
          |\sqrt{4 - x^2} - 0| < \epsilon
        \end{align*}
        所以$\lim\limits_{x \to 2^{-}} \sqrt{4 - x^2} = 0$。

  \item (5) $\lim\limits_{x \to x_0} cos(x) = cos(x_0)$.

        我们有
        \begin{align*}
          |cos(x) - cos(x_0)|
           & = |-2sin(\frac{x + x_0}{2})sin(\frac{x - x_0}{2})|   \\
           & = |2sin(\frac{x + x_0}{2})sin(\frac{x - x_0}{2})|    \\
           & = 2 |sin(\frac{x + x_0}{2})||sin(\frac{x - x_0}{2})| \\
           & \leq 2|sin(\frac{x - x_0}{2})|                       \\
           & \leq 2|\frac{x - x_0}{2}|                            \\
           & = |x - x_0|
        \end{align*}

        于是,对任意的$\epsilon > 0$,取$\delta = \epsilon$,
        则当$|x - x_0| < \delta$时,
        \begin{align*}
          |cos(x) - cos(x_0)| \leq |x - x_0| < \delta = \epsilon
        \end{align*}
        所以$\lim\limits_{x \to x_0} cos(x) = cos(x_0)$。

\end{itemize}

\section*{2}

设函数$f$在点$x_0$的某个空心领域$U^{\circ}(x_0,; \delta^\prime)$
内有定义,$A$为定数。若存在$\epsilon_0 > 0$,对任意的$\delta > 0$,
存在$x \in U^{\circ}(x_0; \delta)$,使得
\begin{align*}
  |f(x) - A| \geq \epsilon_0,
\end{align*}
则称函数$f$当$x$趋于$x_0$时不以$A$为极限,记做
\begin{align*}
  \lim_{x \to x_0} f(x) \neq A
\end{align*}

\section*{3}

$\lim\limits_{x \to x_0} f(x) = A$,
由定义可知,对$\forall \epsilon > 0$,存在$\delta > 0$,
当$0 < |x - x_0| < \delta$时,有
\begin{align*}
  |f(x) - A| < \epsilon
\end{align*}

令
\begin{align*}
  h = x - x_0 \\
  x = x_0 + h
\end{align*}

于是对$\forall \epsilon > 0$,存在$\delta > 0$,
当$0 < |x - x_0| < \delta$时,即$0 < |h| < \delta$,有
\begin{align*}
  |f(x_0 + h) - A| < \epsilon
\end{align*}
由定义可知
\begin{align*}
  \lim\limits_{h \to 0} f(x_0 + h) = A
\end{align*}

\section*{4}

 (1)

利用三角不等式
\begin{align*}
  |a + b| \leq |a| + |b|
\end{align*}

先证明
\begin{align*}
  ||a| - |b|| \leq |a - b|
\end{align*}

因为$a = a - b + b, b = b - a + a$,利用
三角不等式,我们有
\begin{align*}
  |a| \leq |a - b| + |b| \\
  |b| \leq |b - a| + |a|
\end{align*}
移项得
\begin{align*}
  |a| - |b| \leq |a - b| \\
  |b| - |a| \leq |b - a| = |a - b|
\end{align*}
综上
\begin{align*}
  ||a| - |b|| \leq |a - b|
\end{align*}

(2)

已知$\lim\limits_{x \to x_0} f(x) = A$,由定理可知
$\forall \epsilon > 0, \exists \delta > 0$,使得只要$|x - x_0| < \delta$,就有
\begin{align*}
  |f(x) - A| < \epsilon
\end{align*}
由(1)可知,
\begin{align*}
  ||f(x)| - |A|| \leq |f(x) - A| < \epsilon
\end{align*}
所以$\lim\limits_{x \to x_0} |f(x)| = |A|$。

(3) 反过来的情况,个人觉得书中的表述有问题,应该是:\\
$\lim\limits_{x \to x_0} |f(x)| = |A|$,如果$A = 0$,
必能推出$\lim\limits_{x \to x_0} f(x) = A$。
(等于其他值,可以举出反例)

若$A = 0$,由$\lim\limits_{x \to x_0} |f(x)| = 0$可知,
对$\forall \epsilon > 0, \exists \delta > 0$,使得只要
$|x - x_0| < \delta$,有
\begin{align*}
  ||f(x)| - 0| < \epsilon \\
  ||f(x)|| < \epsilon     \\
  |f(x)| < \epsilon
\end{align*}
所以$\lim\limits_{x \to x_0} f(x) = 0$。

\section*{5}

略

\section*{6}

\begin{itemize}
  \item (1)

        \begin{itemize}
          \item $\lim\limits_{x \to 0^{-}} f(x) = -1$;

                $x < 0$,我们有
                \begin{align*}
                  |f(x) - (-1)| & = |\frac{|x|}{x} + 1| \\
                                & = |\frac{-x}{x} + 1|  \\
                                & = |-1 + 1|            \\
                                & = 0
                \end{align*}

                于是对任意$\epsilon > 0$,取$\delta > 0$,只要
                $0 < 0 - x < \delta$,有
                \begin{align*}
                  |f(x) - (-1)| = 0 < \epsilon
                \end{align*}
                所以$\lim\limits_{x \to 0^{-}} f(x) = -1$。

          \item $\lim\limits_{x \to 0^{+}} f(x) = 1$;

                $x > 0$,我们有
                \begin{align*}
                  |f(x) - 1| & = |\frac{|x|}{x} - 1| \\
                             & = |\frac{x}{x} - 1|   \\
                             & = |1 - 1|             \\
                             & = 0
                \end{align*}
                于是对任意$\epsilon > 0$,取$\delta > 0$,只要
                $0 < x - 0 < \delta$,有
                \begin{align*}
                  |f(x) - 1| = 0 < \epsilon
                \end{align*}
                所以$\lim\limits_{x \to 0^{+}} f(x) = 1$。
        \end{itemize}

  \item (2)

        \begin{itemize}
          \item $\lim\limits_{x \to 0^{-}} f(x) = -1$;

                不妨设$-1 < x < 0$,我们有
                \begin{align*}
                  |f(x) - (-1)|
                   & = |[x] + 1| \\
                   & = |-1 + 1|  \\
                   & = 0
                \end{align*}
                于是对任意$\epsilon > 0$,取$1 > \delta > 0$,只要
                $0 < 0 - x < \delta$,有
                \begin{align*}
                  |f(x) - (-1)| = 0 < \epsilon
                \end{align*}
                所以$\lim\limits_{x \to 0^{+}} f(x) = -1$。

          \item $\lim\limits_{x \to 0^{+}} f(x) = 0$;

                不妨设$0 < x < 1$, 我们有
                \begin{align*}
                  |f(x) - 0|
                   & = |[x] - 0| \\
                   & = |[x]|     \\
                   & = 0
                \end{align*}
                于是对任意$\epsilon > 0$,取$1 > \delta > 0$,只要
                $0 < x - 0 < \delta$,有
                \begin{align*}
                  |f(x) - 0| = 0 < \epsilon
                \end{align*}
                所以$\lim\limits_{x \to 0^{+}} f(x) = 0$。
        \end{itemize}

  \item (3)

        \begin{itemize}
          \item  $\lim\limits_{x \to 0^{-}} f(x) = 1$;

                不妨设$x < 0$,我们有
                \begin{align*}
                  |f(x) - 1|
                   & = |1 + x^2 - 1| \\
                   & = x^2
                \end{align*}
                对任意$\epsilon > 0$,
                \begin{align*}
                  x^2 < \epsilon        \\
                  |x| < \sqrt{\epsilon} \\
                  x > - \sqrt{\epsilon}
                \end{align*}
                于是,对任意的$\epsilon$,取$\delta = \sqrt{\epsilon}$,则
                $0 < 0 - x < \sqrt{\epsilon}$,有
                \begin{align*}
                  |f(x) - 1| = x^2 < \epsilon
                \end{align*}
                所以$\lim\limits_{x \to 0^{-}} f(x) = 1$。

          \item $\lim\limits_{x \to 0^{+}} f(x) = 1$;

                不妨设$x > 0$,我们有
                \begin{align*}
                  |f(x) - 1|
                   & = |2^x - 1| \\
                   & = 2^x - 1
                \end{align*}
                对$\epsilon > 0$,
                \begin{align*}
                  2^x - 1 & < \epsilon             \\
                  2^x     & < 1 + \epsilon         \\
                  x       & < \log_2(1 + \epsilon)
                \end{align*}
                于是,$\forall \epsilon > 0, \exists \delta = \log_2(1 + \epsilon)$,则
                $0 < x - 0 < \delta$,有
                \begin{align*}
                  |f(x) - 1| =  2^x - 1 < \epsilon
                \end{align*}
                所以$\lim\limits_{x \to 0^{+}} f(x) = 1$。
        \end{itemize}

\end{itemize}

\section*{7}

已知$\lim\limits_{x \to +\infty} f(x) = A$,
由定义可知,
$\forall \epsilon > 0, \exists M > 0$,使得当$x > M$时,有
\begin{align*}
  |f(x) - A| < \epsilon
\end{align*}
由于当$0 < x < \frac{1}{M}$,有
\begin{align*}
  \frac{1}{x} > M
\end{align*}
于是取
\begin{align*}
  \delta = \frac{1}{M}
\end{align*}
则当$0 < x < \delta$时,有
\begin{align*}
  |f(\frac{1}{x}) - A| < \epsilon
\end{align*}
所以$\lim\limits_{x \to 0^+} f(\frac{1}{x}) = A$。

\section*{8}

\begin{itemize}
  \item $x_0 \in (0, 1)$;

        设任意$\epsilon > 0$。

        如果$x \in (0, 1)$是无理数,我们有
        \begin{align*}
          |R(x) - 0|
           & = |0 - 0| \\
           & = 0
        \end{align*}

        如果$x \in (0, 1)$是有理数,那么可以表示成$\frac{p}{q}$,我们有
        \begin{align*}
          |R(x) - 0|
           & = |R(x)|        \\
           & = |\frac{1}{q}| \\
           & = \frac{1}{q}
        \end{align*}
        所以只需要避开有限个分母$q \leq \frac{1}{\epsilon}$的有理数,就能保证$R(x) = \frac{1}{q} < \epsilon$。

        定义集合
        \begin{align*}
          A := \{q: q \in \mathbb{N}^+, q \leq \frac{1}{\epsilon}\}
        \end{align*}
        是有限集合,取$M = max(A)$。
        因为$x \in (0, 1)$,那么集合
        \begin{align*}
          B : = \{\frac{p}{q}: q, p \in \mathbb{N}^+, p < q \leq M\}
        \end{align*}
        这些是落在$(0,1)$且分母不超过$M$的所有有理数。
        $B$也是有限集合,取$\delta^\prime = min\{|x - x_0|: x \in B, x \neq x_0\}$。
        为了防止出界,取$\delta = min\{\delta^\prime, x_0 - 0, 1 - x_0\}$,
        于是,当$|x - x_0| < \delta$时,
        若$x$是无理数,$R(x) = 0$;
        若$x$是有理数,那么因为$x \notin B$,所以$R(x) < \epsilon$\\
        这两种情况和起来满足:
        \begin{align*}
          |R(x) - 0| < \epsilon
        \end{align*}
        所以$\lim\limits_{x \to x_0} R(x) = 0$。
  \item $x_0 = 0$;

        只需修改$x$的取值范围:
        \begin{align*}
          0 < x - 0 < \delta
        \end{align*}

  \item $x_0 = 1$.

        只需修改$x$的取值范围:
        \begin{align*}
          0 < 1 - x < \delta
        \end{align*}
\end{itemize}



\end{document}