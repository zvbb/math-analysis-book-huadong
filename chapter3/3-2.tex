\documentclass{article}
\usepackage{mathtools} 
\usepackage{fontspec}
\usepackage[UTF8]{ctex}
\usepackage{amsthm}
\usepackage{mdframed}
\usepackage{xcolor}
\usepackage{amssymb}
\usepackage{amsmath}


% 定义新的带灰色背景的说明环境 zremark
\newmdtheoremenv[
  backgroundcolor=gray!10,
  % 边框与背景一致,边框线会消失
  linecolor=gray!10
]{zremark}{说明}

% 通用矩阵命令: \flexmatrix{矩阵名}{元素符号}{行数}{列数}
\newcommand{\flexmatrix}[4]{
  \[
  #1 = \begin{pmatrix}
    #2_{11}     & #2_{12}     & \cdots & #2_{1#4}   \\
    #2_{21}     & #2_{22}     & \cdots & #2_{2#4}   \\
    \vdots      & \vdots      & \ddots & \vdots     \\
    #2_{#31}    & #2_{#32}    & \cdots & #2_{#3#4}
  \end{pmatrix}
  \]
}

% 简化版命令(默认矩阵名为A,元素符号为a): \quickmatrix{行数}{列数}
\newcommand{\quickmatrix}[2]{\flexmatrix{A}{a}{#1}{#2}}

\begin{document}
\title{习题 3.2}
\author{张志聪}
\maketitle

\section*{1}

\begin{itemize}
  \item (6) $\lim\limits_{x \to 4} \frac{\sqrt{1 + 2x} - 3}{\sqrt{x} - 2}$;

        分子有理化

        关键步骤:
        \begin{align*}
          \frac{\sqrt{1 + 2x} - 3}{\sqrt{x} - 2}
           & = \frac{(\sqrt{1 + 2x} - 3)(\sqrt{1 + 2x} + 3)}{(\sqrt{x} - 2)(\sqrt{1 + 2x} + 3)} \\
           & = \frac{2x - 8}{(\sqrt{x} - 2)(\sqrt{1 + 2x} + 3)}                                 \\
           & = \frac{2(\sqrt{x} - 2)(\sqrt{x} + 2)}{(\sqrt{x} - 2)(\sqrt{1 + 2x} + 3)}          \\
           & = \frac{2(\sqrt{x} + 2)}{(\sqrt{1 + 2x} + 3)}                                      \\
        \end{align*}

  \item (8)

        分子分母同时除以$x^{90}$。

        关键步骤:
        \begin{align*}
          \frac{(3x + 6)^{70}(8x - 5)^{20}}{(5x - 1)^{90}}
           & = \frac{(3 + \frac{6}{x})^{70}(8 - \frac{5}{x})^{20}}{(5 - \frac{1}{x})^{90}}
        \end{align*}
        利用四则运算
        \begin{align*}
          \lim\limits_{x \to \infty} (3 + \frac{6}{x})^{70} = 3^{70}
        \end{align*}
\end{itemize}

\section*{3}

\begin{itemize}
  \item (2) $\lim\limits_{x \to x_0} f(x) g(x) = AB$。

        \begin{itemize}
          \item 方法一(直接证);

                $\lim\limits_{x \to x_0} f(x) = A$,由定义可知
                $\forall \epsilon > 0, \exists \delta_1 > 0$,
                当$0 < |x - x_0| < \delta_1$时,有
                \begin{align*}
                  |f(x) - A| < \epsilon
                \end{align*}

                类似地,$\lim\limits_{x \to x_0} g(x) = B$,由定义可知
                $\forall \epsilon > 0, \exists \delta_2 > 0$,
                当$0 < |x - x_0| < \delta_2$时,有
                \begin{align*}
                  |g(x) - B| < \epsilon
                \end{align*}

                综上,取$\delta = \min\{\delta_1, \delta_2\}$,
                当$0 < |x - x_0| < \delta$时,有
                \begin{align*}
                  |f(x)g(x) - AB| < \epsilon|A| + \epsilon|B| + \epsilon^2
                \end{align*}
                因为$|A|, |B|$是定值,$\epsilon$是任意的,
                所以$\lim \limits_{x \to x_0} f(x) g(x) = AB$。

                \begin{zremark}
                  以上证明用到了一个命题:\\
                  设$\epsilon, \delta > 0$,如果$|x - y| < \epsilon, |z - w| < \delta$,
                  那么
                  \begin{align*}
                    |xz - yw| < \epsilon|y| + \delta|w| + \epsilon \delta \\
                    \textbf{或}                                            \\
                    |xz - yw| < \epsilon|z| + \delta|x| + \epsilon \delta
                  \end{align*}

                  证明:

                  记
                  \begin{align*}
                    a : = x - y \\
                    b : = z - w
                  \end{align*}
                  那么
                  \begin{align*}
                    x = y + a \\
                    z = w + b
                  \end{align*}
                  于是
                  \begin{align*}
                    |xz - yw|
                     & = |(y + a)(w + b) - yw|       \\
                     & = |yw + by + aw + ab - yw|    \\
                     & = |by + aw + ab|              \\
                     & \leq |by| + |aw| + |ab|       \\
                     & \leq |b||y| + |a||w| + |a||b| \\
                  \end{align*}
                  又因为
                  \begin{align*}
                    |a| \leq \epsilon \\
                    |b| \leq \delta
                  \end{align*}
                  所以,
                  \begin{align*}
                    |xz - yw| < \epsilon|y| + \delta|w| + \epsilon \delta
                  \end{align*}
                \end{zremark}


          \item 方法二(加减相同项);

                $\lim\limits_{x \to x_0} f(x) = A$,由定义可知
                $\forall \epsilon > 0, \exists \delta_1 > 0$,
                当$0 < |x - x_0| < \delta_1$时,有
                \begin{align*}
                  |f(x) - A| < \epsilon \\
                  A - \epsilon < f(x) < A + \epsilon
                \end{align*}

                类似地,$\lim\limits_{x \to x_0} g(x) = B$,由定义可知
                $\forall \epsilon > 0, \exists \delta_2 > 0$,
                当$0 < |x - x_0| < \delta_2$时,有
                \begin{align*}
                  |g(x) - B| < \epsilon \\
                  B - \epsilon < g(x) < B + \epsilon
                \end{align*}

                我们有
                \begin{align*}
                  |f(x) g(x) - AB|
                   & = |f(x) g(x) - Ag(x) + Ag(x) - AB|    \\
                   & = |g(x)(f(x) - A) + A(g(x) - B)|      \\
                   & \leq |g(x)(f(x) - A)| + |A(g(x) - B)| \\
                   & \leq |g(x)||f(x) - A| + |A||g(x) - B|
                \end{align*}

                由局部有界性可知,$\exists M>0, \delta_3 > 0$,
                当$0 < |x - x_0| < \delta_3$时,有
                \begin{align*}
                  |g(x)| < M
                \end{align*}

                取$\delta = \min\{\delta_1, \delta_2, \delta_3\}$,
                当$0 < |x - x_0| < \delta$时,有
                \begin{align*}
                  |f(x) g(x) - AB|
                   & \leq |g(x)||f(x) - A| + |A||g(x) - B| \\
                   & \leq M\epsilon + |A|\epsilon          \\
                   & = (M + |A|)\epsilon
                \end{align*}
                由于$M + |A|$是定值,$\epsilon$是任意的,
                所以$\lim \limits_{x \to x_0} f(x) g(x) = AB$。

        \end{itemize}
\end{itemize}

\section*{5}

因为$f(x) \geq 0$,由保不等性可得
\begin{align*}
  \lim\limits_{x \to x_0} f(x) = A \geq 0
\end{align*}

关键步骤:
\begin{align*}
  |\sqrt[n]{f(x)} - \sqrt[n]{A}|
   & = |\frac{(\sqrt[n]{f(x)} - \sqrt[n]{A})[(\sqrt[n]{f(x)})^{n - 1} + (\sqrt[n]{f(x)})^{n - 2}(\sqrt[n]{A}) + \cdots + (\sqrt[n]{A})^{n - 1}]}{(\sqrt[n]{f(x)})^{n - 1} + (\sqrt[n]{f(x)})^{n - 2}(\sqrt[n]{A}) + \cdots + (\sqrt[n]{A})^{n - 1}}| \\
   & = \frac{f(x) - A}{(\sqrt[n]{f(x)})^{n - 1} + (\sqrt[n]{f(x)})^{n - 2}(\sqrt[n]{A}) + \cdots + (\sqrt[n]{A})^{n - 1}}                                                                                                                            \\
   & \leq \frac{f(x) - A}{(\sqrt[n]{A})^{n - 1}}
\end{align*}

\section*{6}

提示:与例4证明方式相同,只是$0 < a < 1$时,$a^x$是单减的。

\section*{7}

\begin{itemize}
  \item (1)

        否;举一个反例
        \begin{align*}
          \lim\limits_{x \to 0} x^2 = 0 \\
          \lim\limits_{x \to 0} x = 0
        \end{align*}
        此时$A = B$。

  \item (2)

        是;
        证明:

        取$c = \frac{A + B}{2}$,于是
        \begin{align*}
          B < c < A
        \end{align*}

        $\lim\limits_{x \to x_0} f(x) = A$,
        由保号性可知,存在$\delta_1 > 0$,当$ 0 < |x - x_0| < \delta_1$时,有
        \begin{align*}
          f(x) > c
        \end{align*}

        $\lim\limits_{x \to x_0} g(x) = B$,
        由保号性可知,存在$\delta_2 > 0$,当 $ 0 < |x - x_0| < \delta_2$时,有
        \begin{align*}
          c > g(x)
        \end{align*}

        取$\delta = min\{\delta_1, \delta_2\}$,当 $ 0< |x - x_0| < \delta$时,有
        \begin{align*}
          f(x) > g(x)
        \end{align*}
\end{itemize}

\section*{8}

\begin{itemize}
  \item (3)

        利用$a^n - b^n = (a - b)(a^{n-1} + a^{n-2}b + \cdots + b^{n-1})$。
        设$a = x, b = (-1)$,于是,我们有
        \begin{align*}
          x^3 + 1 = x^3 - (-1)
           & = [x - (-1)](x^2 + x(-1) + (-1)^2) \\
           & = (x + 1)(x^2 - x + 1)
        \end{align*}

        我们有
        \begin{align*}
          \frac{1}{x} - \frac{3}{x^3 + 1}
           & = \frac{(x^2 - x + 1) - 3}{(x+1)(x^2 - x + 1)} \\
           & = \frac{x^2 - x - 2}{(x+1)(x^2 - x + 1)}       \\
           & = \frac{(x+1)(x - 2)}{(x+1)(x^2 - x + 1)}      \\
           & = \frac{x - 2}{x^2 - x + 1}
        \end{align*}
        所以
        \begin{align*}
          \lim\limits_{x \to -1} \frac{1}{x} - \frac{3}{x^3 + 1}
           & = \lim\limits_{x \to -1} \frac{x - 2}{x^2 - x + 1} \\
           & = -1
        \end{align*}

  \item (4)

        提示:

        利用$a^n - b^n = (a - b)(a^{n-1} + a^{n-2}b + \cdots + b^{n-1})$,
        设$a = \sqrt[n]{1 + x}, b = 1$。
\end{itemize}

\section*{9}
略



\end{document}