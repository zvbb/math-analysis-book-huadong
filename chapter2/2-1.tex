\documentclass{article}
\usepackage{mathtools} 
\usepackage{fontspec}
\usepackage[UTF8]{ctex}
\usepackage{amsthm}
\usepackage{mdframed}
\usepackage{xcolor}
\usepackage{amssymb}
\usepackage{amsmath}


% 定义新的带灰色背景的说明环境 zremark
\newmdtheoremenv[
  backgroundcolor=gray!10,
  % 边框与背景一致,边框线会消失
  linecolor=gray!10
]{zremark}{说明}

% 通用矩阵命令: \flexmatrix{矩阵名}{元素符号}{行数}{列数}
\newcommand{\flexmatrix}[4]{
  \[
  #1 = \begin{pmatrix}
    #2_{11}     & #2_{12}     & \cdots & #2_{1#4}   \\
    #2_{21}     & #2_{22}     & \cdots & #2_{2#4}   \\
    \vdots      & \vdots      & \ddots & \vdots     \\
    #2_{#31}    & #2_{#32}    & \cdots & #2_{#3#4}
  \end{pmatrix}
  \]
}

% 简化版命令(默认矩阵名为A,元素符号为a): \quickmatrix{行数}{列数}
\newcommand{\quickmatrix}[2]{\flexmatrix{A}{a}{#1}{#2}}

\begin{document}
\title{习题 2.1}
\author{张志聪}
\maketitle

\section*{2}

只写关键步骤

(3) $\lim\limits_{n \to \infty} \frac{n!}{n^n} = 0$。

我们有
\begin{align*}
  |\frac{n!}{n^n} - 0|
   & = |\frac{n!}{n^n}|                                                     \\
   & = \frac{1 \times 2 \times \cdots n}{n \times n \times \cdots \times n} \\
   & < \frac{1}{n}
\end{align*}

(4) $\lim\limits_{n \to \infty} sin \frac{\pi}{n} = 0$。

不妨令$n > 2$,当$0 < x < \frac{\pi}{2}$时,我们有
\begin{align*}
  0 < sin x < x < tan x
\end{align*}

于是,我们有
\begin{align*}
  |sin \frac{\pi}{n} - 0|
   & = sin \frac{\pi}{n} < \frac{\pi}{n} < \epsilon
\end{align*}

(5) $\lim\limits_{n \to \infty} \frac{n}{a^n} = 0$。

令$h = a - 1 > 0$,
于是,我们有
\begin{align*}
  |\frac{n}{a^n}|
   & = \frac{n}{a^n}       \\
   & = \frac{n}{(1 + h)^n} \\
   & \leq \frac{n}{1 + nh}
\end{align*}

\section*{9}

\begin{itemize}
  \item (1) $\lim\limits_{n \to \infty} \sqrt{n + 1} - \sqrt{n} = 0$。

        \begin{align*}
          |\sqrt{n + 1} - \sqrt{n} - 0|
           & = \sqrt{n + 1} - \sqrt{n}                                                            \\
           & = \frac{(\sqrt{n + 1} - \sqrt{n})(\sqrt{n + 1} + \sqrt{n})}{\sqrt{n + 1} + \sqrt{n}} \\
           & = \frac{1}{\sqrt{n + 1} + \sqrt{n}}
        \end{align*}
        因为
        \begin{align*}
          \sqrt{n + 1} > \sqrt{n}
        \end{align*}
        于是,我们有
        \begin{align*}
          \frac{1}{\sqrt{n + 1} + \sqrt{n}} < \frac{1}{2 \sqrt{n}}
        \end{align*}

  \item (2) $\lim\limits_{n \to infty} \frac{1 + 2 + 3 + \cdots + n}{n^3} = 0$。

        \begin{align*}
          |\frac{1 + 2 + 3 + \cdots + n}{n^3} - 0|
           & = \frac{\frac{n(n+1)}{2}}{n^3}      \\
           & = \frac{n^2 + n}{2n^3}              \\
           & = \frac{n^2}{2n^3} + \frac{n}{2n^3} \\
           & = \frac{1}{2n} + \frac{1}{2n^2}     \\
           & \leq \frac{1}{2n} + \frac{1}{2n}    \\
           & = \frac{1}{n} < \epsilon
        \end{align*}

  \item (3)

        对于偶数,$N$是好找的。对于奇数,我们有
        \begin{align*}
          |\frac{\sqrt{n^2 + n}}{n} - 1|
           & = |\frac{\sqrt{n^2 + n}}{\sqrt{n^2}} - 1| \\
           & = |\sqrt{1 + \frac{1}{n}} - 1|            \\
           & = \sqrt{1 + \frac{1}{n}} - 1 < \epsilon
        \end{align*}


\end{itemize}

\end{document}