\documentclass{article}
\usepackage{mathtools} 
\usepackage{fontspec}
\usepackage[UTF8]{ctex}
\usepackage{amsthm}
\usepackage{mdframed}
\usepackage{xcolor}
\usepackage{amssymb}
\usepackage{amsmath}


% 定义新的带灰色背景的说明环境 zremark
\newmdtheoremenv[
  backgroundcolor=gray!10,
  % 边框与背景一致,边框线会消失
  linecolor=gray!10
]{zremark}{说明}

% 通用矩阵命令: \flexmatrix{矩阵名}{元素符号}{行数}{列数}
\newcommand{\flexmatrix}[4]{
  \[
  #1 = \begin{pmatrix}
    #2_{11}     & #2_{12}     & \cdots & #2_{1#4}   \\
    #2_{21}     & #2_{22}     & \cdots & #2_{2#4}   \\
    \vdots      & \vdots      & \ddots & \vdots     \\
    #2_{#31}    & #2_{#32}    & \cdots & #2_{#3#4}
  \end{pmatrix}
  \]
}

% 简化版命令(默认矩阵名为A,元素符号为a): \quickmatrix{行数}{列数}
\newcommand{\quickmatrix}[2]{\flexmatrix{A}{a}{#1}{#2}}

\begin{document}
\title{习题 2.2}
\author{张志聪}
\maketitle

\section*{1}

\begin{itemize}
  \item (3)

        它的偶数项组成的子列为
        \begin{align*}
          \frac{2^n + 3^n}{3^{n + 1} - 2^{n + 1}}
           & = \frac{\frac{1}{3}\left(\frac{2}{3}\right)^n + \frac{1}{3}}{1 - \left(\frac{2}{3}\right)^{n + 1}}
        \end{align*}
        我们有
        \begin{align*}
          \lim\limits_{n \to \infty} \frac{1}{3}\left(\frac{2}{3}\right)^n + \frac{1}{3} = \frac{1}{3} \\
          \lim\limits_{n \to \infty} 1 - \left(\frac{2}{3}\right)^{n + 1} = 1
        \end{align*}
        利用定理2.7(四则运算)可知
        \begin{align*}
          \lim\limits_{n \to \infty}  \frac{\frac{1}{3}\left(\frac{2}{3}\right)^n + \frac{1}{3}}{1 - \left(\frac{2}{3}\right)^{n + 1}}
           & = \frac{1}{3}
        \end{align*}

        它的奇数项组成的子列为
        \begin{align*}
          \frac{3^n - 2^n}{2^{n + 1} + 3^{n + 1}}
           & = \frac{\frac{1}{3} - \frac{1}{3}\left(\frac{2}{3}\right)^{n}}{(\frac{2}{3})^{n + 1} + 1}
        \end{align*}

        我们有
        \begin{align*}
          \lim\limits_{n \to \infty} \frac{1}{3} - \frac{1}{3}\left(\frac{2}{3}\right)^{n} = \frac{1}{3} \\
          \lim\limits_{n \to \infty} (\frac{2}{3})^{n + 1} + 1 = 1
        \end{align*}
        利用定理2.7(四则运算)可知
        \begin{align*}
          \frac{\frac{1}{3} - \frac{1}{3}\left(\frac{2}{3}\right)^{n}}{(\frac{2}{3})^{n + 1} + 1}
           & = \frac{1}{3}
        \end{align*}

        综上,数列满足偶数项极限与奇数项极限相等,利用第一节例8可得,极限为$\frac{1}{3}$。
\end{itemize}

\section*{4}

\begin{itemize}
  \item (1)

        关键点
        \begin{align*}
          \frac{1}{n(n + 1)} = \frac{1}{n} - \frac{1}{n + 1}
        \end{align*}

  \item (2)

        我们有
        \begin{align*}
          \sqrt{2} \sqrt[4]{2} \sqrt[8]{2}\cdots \sqrt[2^n]{2}
           & = 2^{\frac{1}{2}} 2^{\frac{1}{4}} 2^{\frac{1}{8}} \cdots 2^{\frac{1}{2^n}} \\
           & = 2^{1 - \frac{1}{2^n}}                                                    \\
           & = \frac{2}{\sqrt[2^n]{2}}
        \end{align*}
        又因为
        \begin{align*}
          \lim\limits_{n \to \infty} \sqrt[2^n]{2} = 1
        \end{align*}
        (证明与第一节例5相同)
        利用极限的四则运算(定理2.7),我们有
        \begin{align*}
          \lim\limits_{n \to \infty} \sqrt{2} \sqrt[4]{2} \sqrt[8]{2}\cdots \sqrt[2^n]{2}
          = 2
        \end{align*}

  \item (3)

        这道题要找到求和公式。
        可以观察出分子是等差数列,分母是等比数列,
        这种大部分都是采用错位相减的方式。
        \begin{align*}
          S_n = \frac{1}{2} + \frac{3}{2^2} + \cdots + \frac{2n - 1}{2^n} \\
          \frac{1}{2}S_n = \frac{1}{2^2} + \frac{3}{2^3} + \cdots + \frac{2n - 1}{2^{n+1}}
        \end{align*}
        于是,两式相减
        \begin{align*}
          \frac{1}{2}S_n
           & = \frac{1}{2} + \frac{2}{2^2} + \frac{2}{2^3} + \cdots + \frac{2}{2^n} - \frac{2n - 1}{2^{n+1}}    \\
           & = \frac{1}{2} + \frac{1}{2}+ \frac{1}{2^2} + \cdots + \frac{1}{2^{n - 1}} - \frac{2n - 1}{2^{n+1}}
        \end{align*}
        其中,
        \begin{align*}
          \frac{1}{2}+ \frac{1}{2^2} + \cdots + \frac{1}{2^{n - 1}}
        \end{align*}
        是等比数列,其和为$1 - \frac{1}{2^{n - 1}}$。
        于是,我们有
        \begin{align*}
          \frac{1}{2}S_n & = \frac{1}{2} + 1 - \frac{1}{2^{n - 1}} - \frac{2n - 1}{2^{n+1}} \\
          S_n            & = 3 - \frac{1}{2^{n - 2}} - \frac{2n - 1}{2^{n}}
        \end{align*}
        所以
        \begin{align*}
          \lim\limits_{n \to \infty} \frac{1}{2} + \frac{3}{2^2} + \cdots + \frac{2n - 1}{2^n}
           & = \lim\limits_{n \to \infty} 3 - \frac{1}{2^{n - 2}} - \frac{2n - 1}{2^{n}} \\
           & = 3
        \end{align*}

  \item (4)

        我们有
        \begin{align*}
          \sqrt[n]{1 - \frac{1}{n}}
           & = \sqrt[n]{\frac{n - 1}{n}}           \\
           & = \frac{\sqrt[n]{n - 1}}{\sqrt[n]{n}}
        \end{align*}

        当$n \geq 2$时,我们有
        \begin{align*}
          1 \leq \sqrt[n]{n - 1} \leq \sqrt[n]{n}
        \end{align*}
        利用迫敛性和例2可得
        \begin{align*}
          \lim\limits_{n \to \infty} \sqrt[n]{n - 1} = 1
        \end{align*}
        利用极限的四则运算,我们有
        \begin{align*}
          \lim\limits_{n \to infty} \sqrt[n]{1 - \frac{1}{n}}
           & = \frac{\lim\limits_{n \to \infty} \sqrt[n]{n - 1}}{\lim\limits_{n \to \infty} \sqrt[n]{n}} \\
           & = 1
        \end{align*}

  \item (5)

        我们有
        \begin{align*}
          \frac{1}{n^2} + \frac{1}{(n + 1)^2} + \cdots + \frac{1}{(2n)^2}
           & < \frac{1}{n^2} + \frac{1}{n^2} + \cdots + \frac{1}{n^2} \\
           & = \frac{n + 1}{n^2}
        \end{align*}
        又因为
        \begin{align*}
          0 < \frac{1}{n^2} + \frac{1}{(n + 1)^2} + \cdots + \frac{1}{(2n)^2} \\
          \lim\limits_{n \to \infty} \frac{n + 1}{n^2} = 0
        \end{align*}
        由迫敛性可得
        \begin{align*}
          \lim\limits_{n \to \infty} \left(\frac{1}{n^2} + \frac{1}{(n + 1)^2} + \cdots + \frac{1}{(2n)^2}\right)
           & = 0
        \end{align*}

  \item (6)

        我们有
        \begin{align*}
          \frac{n}{\sqrt{n^2 + n}} \leq \frac{1}{\sqrt{n^2 + 1}} + \frac{1}{\sqrt{n^2 + 2}} + \cdots + \frac{1}{\sqrt{n^2 + n}} \\
          \frac{1}{\sqrt{n^2 + 1}} + \frac{1}{\sqrt{n^2 + 2}} + \cdots + \frac{1}{\sqrt{n^2 + n}} \leq \frac{n}{\sqrt{n^2 + 1}}
        \end{align*}
        又因为
        \begin{align*}
          \lim\limits_{n \to \infty} \frac{n}{\sqrt{n^2 + n}}
           & = \lim\limits_{n \to \infty} \sqrt{\frac{n^2}{n^2 + n}}       \\
           & = \lim\limits_{n \to \infty} \sqrt{\frac{1}{1 + \frac{1}{n}}} \\
           & = 1
        \end{align*}
        同理
        \begin{align*}
          \lim\limits_{n \to \infty} \frac{n}{\sqrt{n^2 + 1}} = 1
        \end{align*}
        由夹逼定理可得
        \begin{align*}
          \lim\limits_{n \to \infty} \frac{1}{\sqrt{n^2 + 1}} + \frac{1}{\sqrt{n^2 + 2}} + \cdots + \frac{1}{\sqrt{n^2 + n}} = 1
        \end{align*}

\end{itemize}

\section*{8}

\begin{itemize}
  \item (1) $\lim\limits_{n \to \infty} \frac{1}{2} \cdot \frac{3}{4} \cdot \cdots \cdot \frac{2n - 1}{2n}$

        因为$n^2 \geq n^2 - 1 = (n - 1)(n + 1)$,于是
        我们有
        \begin{align*}
           & \frac{1}{2} \cdot \frac{3}{4} \cdot \cdots \cdot \frac{2n -1}{2n}                                              \\
           & \leq \frac{1}{\sqrt{3}} \cdot \frac{3}{\sqrt{3}\sqrt{5}} \cdot \cdots \cdot \frac{2n - 1}{\sqrt{2n - 1}{2n+1}} \\
           & = \frac{1}{\sqrt{2n + 1}}
        \end{align*}

        又因为
        \begin{align*}
          0 \leq \frac{1}{2} \cdot \frac{3}{4} \cdot \cdots \cdot \frac{2n - 1}{2n} \\
          \lim\limits_{n \to \infty} \frac{1}{\sqrt{2n + 1}} = 0
        \end{align*}
        由夹逼定理可得
        \begin{align*}
          \lim\limits_{n \to \infty} \frac{1}{2} \cdot \frac{3}{4} \cdot \cdots \cdot \frac{2n - 1}{2n} = 0
        \end{align*}

  \item (2)

        先考虑$\sum\limits_{p=1}^n p!$,增大部分项的值(前$n - 2$项),我们有
        \begin{align*}
          n ! \leq \sum\limits_{p=1}^n p! & \leq (n-2)(n-2)! + (n-1)! + n!         \\
                                          & = (n - 1)(n-2)! - (n-2)! + (n-1)! + n! \\
                                          & = (n-1)! - (n - 2)! + (n-1)! + n!      \\
                                          & \leq 2(n-1)! + n!
        \end{align*}
        因为
        \begin{align*}
          \lim\limits_{n \to \infty} \frac{n!}{n!} = 1 \\
          \lim\limits_{n \to \infty} \frac{2(n-1)! + n!}{n!} = 1
        \end{align*}
        由迫敛性可得
        \begin{align*}
          \lim\limits_{n \to \infty} \frac{\sum\limits_{p=1}^n p!}{n!} = 1
        \end{align*}

  \item (3)

        我们有
        \begin{align*}
          (n+1)^\alpha - n^\alpha & = n^\alpha [(1 + \frac{1}{n})^\alpha - 1]
        \end{align*}
        因为$1 + \frac{1}{n} > 1$,且$0 \leq \alpha < 1$,所以
        \begin{align*}
          (1 + \frac{1}{n})^\alpha < 1 + \frac{1}{n}
        \end{align*}
        所以
        \begin{align*}
          (n+1)^\alpha - n^\alpha & = n^\alpha [(1 + \frac{1}{n})^\alpha - 1] \\
                                  & \leq n^\alpha [1 + \frac{1}{n} - 1]       \\
                                  & = n^\alpha \frac{1}{n}                    \\
                                  & = n^{\alpha - 1}                          \\
                                  & = \frac{1}{n^{1 - \alpha}}
        \end{align*}
        又因为
        \begin{align*}
          0 < (n+1)^\alpha - n^\alpha \\
          \lim\limits_{n \to \infty} \frac{1}{n^{1 - \alpha}} = 0
        \end{align*}
        由迫敛性可知
        \begin{align*}
          \lim\limits_{n \to \infty} (n+1)^\alpha - n^\alpha = 0
        \end{align*}
  \item (4)

        我们有
        \begin{align*}
           & (1+\alpha)(1+\alpha^2)\cdots(1+\alpha^{2^n})                              \\
           & = \frac{(1-\alpha)(1+\alpha)(1+\alpha^2)\cdots(1+\alpha^{2^n})}{1-\alpha} \\
           & = \frac{1 - \alpha^{2^{n+1}}}{1 - \alpha}                                 \\
           & = \frac{1}{1 - \alpha} - \frac{\alpha^{2^{n+1}}}{1 - \alpha}
        \end{align*}
        因为$|\alpha| < 1$,于是
        \begin{align*}
          0 \leq \alpha^{2^{n + 1}} < \alpha^n
        \end{align*}
        又因为
        \begin{align*}
          \lim\limits_{n \to \infty} \alpha^n = 0
        \end{align*}
        由迫敛性可知
        \begin{align*}
          \lim\limits_{n \to \infty} \alpha^{2^{n + 1}} = 0
        \end{align*}
        利用极限的四则运算可得
        \begin{align*}
          \lim\limits_{n \to \infty} (1+\alpha)(1+\alpha^2)\cdots(1+\alpha^{2^n})
           & = \frac{1}{1 - \alpha} - 0 \\
           & = \frac{1}{1 - \alpha}
        \end{align*}
\end{itemize}

\section*{9}

设
\begin{align*}
  a = max\{a_1, a_2, \cdots, a_m\}
\end{align*}

我们有
\begin{align*}
  \sqrt[n]{a^n} = a
  \leq
  \sqrt[n]{a_1^n + a_2^n + \cdots + a_m^n}
  \leq
  \sqrt[n]{a^n + a^n + \cdots + a^n}
  = \sqrt[n]{m a^n} = \sqrt[n]{m}a
\end{align*}
又因为
\begin{align*}
  \lim\limits_{n \to \infty} \sqrt[n]{m}a
   & = \lim\limits_{n \to \infty} \sqrt[n]{m} \lim\limits_{n \to \infty} a \\
   & = 1 \cdot a = a
\end{align*}
由迫敛性可得
\begin{align*}
  \lim\limits_{n \to \infty} \sqrt[n]{a_1^n + a_2^n + \cdots + a_m^n} = a
\end{align*}

\end{document}