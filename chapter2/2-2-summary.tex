\documentclass{article}
\usepackage{mathtools} 
\usepackage{fontspec}
\usepackage[UTF8]{ctex}
\usepackage{amsthm}
\usepackage{mdframed}
\usepackage{xcolor}
\usepackage{amssymb}
\usepackage{amsmath}


% 定义新的带灰色背景的说明环境 zremark
\newmdtheoremenv[
  backgroundcolor=gray!10,
  % 边框与背景一致,边框线会消失
  linecolor=gray!10
]{zremark}{说明}

% 通用矩阵命令: \flexmatrix{矩阵名}{元素符号}{行数}{列数}
\newcommand{\flexmatrix}[4]{
  \[
  #1 = \begin{pmatrix}
    #2_{11}     & #2_{12}     & \cdots & #2_{1#4}   \\
    #2_{21}     & #2_{22}     & \cdots & #2_{2#4}   \\
    \vdots      & \vdots      & \ddots & \vdots     \\
    #2_{#31}    & #2_{#32}    & \cdots & #2_{#3#4}
  \end{pmatrix}
  \]
}

% 简化版命令(默认矩阵名为A,元素符号为a): \quickmatrix{行数}{列数}
\newcommand{\quickmatrix}[2]{\flexmatrix{A}{a}{#1}{#2}}

\begin{document}
\title{总结 2.2}
\author{张志聪}
\maketitle

\section*{1}

\begin{zremark}
  核心:

  1.利用极限的四则运算,计算极限。

  2.利用迫敛性计算极限。

  其中,迫敛性技巧性比较强,要对数列中的项需要进行缩放,
  常见缩放方式有哪些?
\end{zremark}

常见题型:累加或累乘的形式表示数列的项
\begin{itemize}
  \item 累加(或累乘)形式表示$a_n$,用特定项(最小项或最大项或某一项)代替所有项(或部分项),进行缩放。
        如:习题4-(5),习题8-(2)
  \item 累乘利用$n^ \geq n^2 - 1 = (n - 1)(n + 1)$不等式,对分母进行拆分,达到与分子约分的目的。
        如:习题8-(1)
  \item 加项,使得$a_n$可以化简。如:习题8-(4)
\end{itemize}


\end{document}