\documentclass{article}
\usepackage{mathtools} 
\usepackage{fontspec}
\usepackage[UTF8]{ctex}
\usepackage{amsthm}
\usepackage{mdframed}
\usepackage{xcolor}
\usepackage{amssymb}
\usepackage{amsmath}
\usepackage[most]{tcolorbox}


% 定义新的带灰色背景的说明环境 zremark
\newmdtheoremenv[
  backgroundcolor=gray!10,
  % 边框与背景一致,边框线会消失
  linecolor=gray!10
]{zremark}{说明}

\newtcolorbox{zthk}{
  colback=blue!5,
  colframe=blue!40!black,
  fonttitle=\bfseries,
  title=思维过程,
  fontupper=\itshape,
  separator sign={.},
  before skip=10pt,
  after skip=10pt,
  breakable
}

% 通用矩阵命令: \flexmatrix{矩阵名}{元素符号}{行数}{列数}
\newcommand{\flexmatrix}[4]{
  \[
  #1 = \begin{pmatrix}
    #2_{11}     & #2_{12}     & \cdots & #2_{1#4}   \\
    #2_{21}     & #2_{22}     & \cdots & #2_{2#4}   \\
    \vdots      & \vdots      & \ddots & \vdots     \\
    #2_{#31}    & #2_{#32}    & \cdots & #2_{#3#4}
  \end{pmatrix}
  \]
}

% 简化版命令(默认矩阵名为A,元素符号为a): \quickmatrix{行数}{列数}
\newcommand{\quickmatrix}[2]{\flexmatrix{A}{a}{#1}{#2}}

\begin{document}
\title{习题 2.3}
\author{张志聪}
\maketitle

\section*{1}

\begin{itemize}
  \item (1) $\lim\limits_{n \to \infty} \left(1 - \frac{1}{n}\right)^{n}$;

        我们有
        \begin{align*}
          \left(1 - \frac{1}{n}\right)^{n}
           & = \left(\frac{n - 1}{n}\right)^{n}                                                 \\
           & = \frac{1}{\left(\frac{n}{n - 1}\right)^{n}}                                       \\
           & = \frac{1}{\left(\frac{n-1+1}{n - 1}\right)^{n}}                                   \\
           & = \frac{1}{\left(1 + \frac{1}{n - 1}\right)^{n}}                                   \\
           & = \frac{1}{\left(1 + \frac{1}{n - 1}\right)^{n-1}\left(1 + \frac{1}{n - 1}\right)}
        \end{align*}
        设
        \begin{align*}
          a_n = \left(1 + \frac{1}{n}\right)^{n} \\
          b_n = \left(1 + \frac{1}{n - 1}\right)^{n-1} \ (n \geq 2)
        \end{align*}
        于是可得,
        \begin{align*}
          b_2 = a_1 \\
          b_3 = a_2 \\
          \vdots    \\
          b_n = a_{n-1}
          \vdots
        \end{align*}
        由数列改变有限项,极限不变可得
        \begin{align*}
          \lim\limits_{n \to \infty} \left(1 + \frac{1}{n - 1}\right)^{n-1}
           & = \lim\limits_{n \to \infty} \left(1 + \frac{1}{n}\right)^{n} \\
           & = e
        \end{align*}

        由极限的四则运算可得
        \begin{align*}
          \lim\limits_{n \to \infty} \left(1 - \frac{1}{n}\right)^{n}
           & = \lim\limits_{n \to \infty} \frac{1}{\left(1 + \frac{1}{n - 1}\right)^{n-1}\left(1 + \frac{1}{n - 1}\right)} \\
           & = \frac{1}{e \cdot 1} = \frac{1}{e}
        \end{align*}
  \item (3) $\lim\limits_{n \to \infty} \left(1 + \frac{1}{n + 1}\right)^n$;

        我们有
        \begin{align*}
          \left(1 + \frac{1}{n + 1}\right)^n
           & = \frac{\left(1 + \frac{1}{n + 1}\right)^{n + 1}}{\left(1 + \frac{1}{n + 1}\right)}
        \end{align*}
        设
        \begin{align*}
          a_n = \left(1 + \frac{1}{n}\right)^n \\
          b_n = \left(1 + \frac{1}{n + 1}\right)^{n + 1}
        \end{align*}
        可得数列$(b_n)_{n = 1}^{\infty}$是$(a_n)_{n = 1}^{\infty}$的子列,于是极限相同,
        所以
        \begin{align*}
          \lim_{n \to \infty} \left(1 + \frac{1}{n + 1}\right)^{n + 1}
           & = \lim_{n \to \infty} \left(1 + \frac{1}{n}\right)^n \\
           & = e
        \end{align*}

        由极限的四则运算可得
        \begin{align*}
          \lim\limits_{n \to \infty} \left(1 + \frac{1}{n + 1}\right)^n
           & = \lim\limits_{n \to \infty} \frac{\left(1 + \frac{1}{n + 1}\right)^{n + 1}}{\left(1 + \frac{1}{n + 1}\right)} \\
           & = \frac{e}{1}                                                                                                  \\
           & = e
        \end{align*}

  \item (5) $\lim\limits_{n \to \infty} \left(1 + \frac{1}{n^2}\right)^n$ .

        我们有
        \begin{align*}
          \left(1 + \frac{1}{n^2}\right)^n
           & = \left[\left(1 + \frac{1}{n^2}\right)^{n^2}\right]^\frac{1}{n}
        \end{align*}
        由$\left(1 + \frac{1}{n^2}\right)^{n^2}$是
        $\left(1 + \frac{1}{n}\right)^{n}$的子列可得
        \begin{align*}
          \lim\limits_{n \to \infty} \left(1 + \frac{1}{n^2}\right)^{n^2} = e
        \end{align*}

        利用$\S 2$习题10可得
        \begin{align*}
          \lim\limits_{n \to \infty} \left(1 + \frac{1}{n^2}\right)^n
           & = \lim\limits_{n \to \infty} \left[\left(1 + \frac{1}{n^2}\right)^{n^2}\right]^\frac{1}{n} \\
           & = 1
        \end{align*}

\end{itemize}

\section*{3}

\begin{itemize}
  \item (1)

        易得$2$是数列的上界(对$n$进行归纳即可)。

        证明数列是单增的。对任意$n$,我们有
        \begin{align*}
          a_{n + 1} = \sqrt{2 a_n} \\
          a_{n + 1}^2 = 2 a_n
        \end{align*}
        假设$a_n > a_{n + 1}$,即$a_{n + 1} < a_n < 2$可得
        \begin{align*}
          a_{n + 1}^2 < 2 a_n
        \end{align*}
        存在矛盾,假设不成立,于是$a_n \leq a_{n + 1}$。

        由单调有界定理,数列有极限,记为$a$。由于
        \begin{align*}
          a_{n + 1}^2 = 2 a_n
        \end{align*}
        对上式两边取极限得
        \begin{align*}
          a^2 = 2 a
        \end{align*}
        解得$a = 0$或$a = 2$.

        由数列极限的保不等式性,$a = 0$是不可能的,故极限是$2$。

  \item (2)

        先证明是单调递增的。

        设
        \begin{align*}
          b_1       & = 0              \\
          b_{n + 1} & = \sqrt{c + b_n}
        \end{align*}
        于是
        \begin{align*}
          b_{n} & < a_n
        \end{align*}
        (对$n$进行归纳即可)

        且我们有
        \begin{align*}
          a_n & = \sqrt{c + b_n}
        \end{align*}

        所以
        \begin{align*}
          a_{n + 1} & = \sqrt{c + a_n} \\
                    & > \sqrt{c + b_n} \\
                    & = a_n
        \end{align*}
        综上,数列$(a_n)_{n = 1}^\infty$是单调递增的。

        证明数列的上界为$\frac{1 + \sqrt{1 + 4c}}{2}$。
        \begin{zthk}
          题目说求极限,记为$a$,通过递归公式
          \begin{align*}
            a_{n + 1} & = \sqrt{c + a_n}
          \end{align*}
          对等式两边求极限
          \begin{align*}
            a^2 = c + a \\
            a^2 -a - c = 0
          \end{align*}
          解为$\frac{1 \pm \sqrt{1 + 4c}}{2}$,由于$a_1 = \sqrt{c}>0$且由数列递增可知,$a_n > \sqrt{c}$,
          由极限保不等式性可知,
          解不会是$\frac{1 - \sqrt{1 + 4c}}{2}$。

          由于单调有界定理可知,极限与数列的上确界是相等的,我们可以把上界设为$\frac{1 + \sqrt{1 + 4c}}{2}$。
        \end{zthk}
        (以上是思考过程,不用写到证明过程中)

        证明,数列$(a_n)_{n = 1}^\infty$的上界是$\frac{1 + \sqrt{1 + 4c}}{2}$。

        对$n$进行归纳。

        $n = 1$时,$a_1 = \sqrt{c} < \frac{1 + \sqrt{1 + 4c}}{2}$。

        归纳假设$n = k$时,$a_n \leq \frac{1 + \sqrt{1 + 4c}}{2}$。

        $n = k + 1$时,
        \begin{align*}
          a_{k + 1} & = \sqrt{c + a_n}                            \\
                    & \leq \sqrt{c + \frac{1 + \sqrt{1 + 4c}}{2}}
        \end{align*}
        因为$\frac{1 + \sqrt{1 + 4c}}{2}$是
        \begin{align*}
          \sqrt{c + x} = x
        \end{align*}
        的解,所以
        \begin{align*}
          a_{k + 1} & \leq \sqrt{c + \frac{1 + \sqrt{1 + 4c}}{2}} \\
                    & = \frac{1 + \sqrt{1 + 4c}}{2}
        \end{align*}

        归纳完成。

        由单调有界定理可得,数列$(a_n)_{n = 1}^\infty$收敛,
        然后可得$\lim\limits_{n \to \infty} a_n = \frac{1 + \sqrt{1 + 4c}}{2}$(运算方式与思考过程一样,不做赘述)。


  \item (3)

        我们有
        \begin{align*}
          \frac{a_{n + 1}}{a_n}
           & = \frac{\frac{c^{n + 1}}{(n + 1)!}}{\frac{c^n}{n!}} \\
           & = \frac{c}{n + 1}
        \end{align*}
        于是,取$N > c, N \in \mathbb{N}^+$,对任意$n \geq N$都有
        \begin{align*}
          \frac{a_{n + 1}}{a_n} < 1
        \end{align*}
        即$a_n$从第$N$项开始,单调递减。

        又因为,任意$n$都有
        \begin{align*}
          a_n > 0
        \end{align*}

        由单调有界定理可知,${a_n}$收敛

        设$\lim\limits_{n \to \infty} a_n = a$,由于
        \begin{align*}
          a_{n+1} = \frac{n + 1}{c} a_n
        \end{align*}
        对上式两边取极限得$a = a \cdot 0 = 0$。
        故有
        \begin{align*}
          \lim\limits_{n \to \infty} a_n = 0
        \end{align*}
\end{itemize}

\section*{5}

我们有
\begin{align*}
  |a_n - a_m|
   & = \frac{1}{(m + 1)^2} + \frac{1}{(m + 2)^2} + \cdots + \frac{1}{n^2}                                               \\
   & < \frac{1}{m(m + 1)} + \frac{1}{(m + 1)(m + 2)} + \cdots + \frac{1}{(n - 1)n}                                      \\
   & = (\frac{1}{m} - \frac{1}{m + 1}) + (\frac{1}{m + 1} - \frac{1}{m + 2}) + \cdots + (\frac{1}{n - 1} - \frac{1}{n}) \\
   & = \frac{1}{m} - \frac{1}{n}                                                                                        \\
   & < \frac{1}{m}
\end{align*}
对任意的$\epsilon > 0$,取$N = \frac{1}{\epsilon}$,则对一切$n > m > N$,有
\begin{align*}
  |a_n - a_m| < \epsilon
\end{align*}
所以,数列满足柯西条件。

\section*{6}
提示:
子列收敛,则说明子列有界,利用$\{a_n\}$的单调性可得,$\{a_n\}$也是有界的。

\section*{7}

易得$\{a_n\}$单调递减且$0$是$\{a_n\}$下界,所以$\{a_n\}$收敛,记为$a$。

\begin{itemize}
  \item (1) 方法一

        因为$a_n \geq 0$,由保不等式性可知
        \begin{align*}
          a \geq 0
        \end{align*}

        取$1 < l^\prime < l$,由保号性可知,存在$N$,使得只要$n \geq N$都有
        \begin{align*}
          \frac{a_n}{a_{n + 1}} > l^\prime \\
          a_n > l^\prime a_{n + 1}
        \end{align*}

        对上式两边取极限,并由保不等式性得
        \begin{align*}
          a \geq a l^\prime      \\
          0 \geq a(l^\prime - 1) \\
          0 \geq a
        \end{align*}

        综上,$a = 0$,即:
        \begin{align*}
          \lim_{n \to \infty} a_n = 0
        \end{align*}

  \item 方法二

        我们有
        \begin{align*}
          a_n = \frac{a_n}{a_{n + 1}} a_{n + 1}
        \end{align*}
        两边取极限
        \begin{align*}
          a         & = l \cdot a \\
          (l - 1) a & = 0
        \end{align*}
        因为$(l - 1) \neq 0$,于是$a = 0$。

  \item 方法三

        设$1 < t < l$,由保号性可知,存在$N$,使得只要$n \geq N$有
        \begin{align*}
          \frac{a_n}{a_{n + 1}} > t
        \end{align*}
        即
        \begin{align*}
          0 < \frac{a_{n + 1}}{a_n} < \frac{1}{t}
        \end{align*}
        我们有
        \begin{align*}
          0 < a_n = a_N \cdot \frac{a_{N + 1}}{a_N} \cdot \frac{a_{N + 2}}{a_{N + 1}} \cdots \frac{a_n}{a_{n - 1}}
          < a_N (\frac{1}{t})^{n - N}
        \end{align*}

        由迫敛性可知,
        \begin{align*}
          \lim\limits_{n \to \infty} a_n = 0
        \end{align*}

\end{itemize}

\section*{9}

 (1)

先证明题目中的不等式是如何推导出来的。
\begin{align*}
  b^{n + 1} & = (a + (b - a))^{n + 1}                                                                                        \\
            & = a^{n + 1} + C_{n+1}^1 a^{n} (b - a) + C_{n+1}^2 a^{n - 1} (b - a)^2 + \cdots + C_{n+1}^{n + 1} (b-a)^{n + 1} \\
            & > a^{n + 1} + C_{n+1}^1 a^{n} (b - a)                                                                          \\
            & = a^{n + 1} + (n + 1) a^{n} (b - a)
\end{align*}
移项得
\begin{align*}
  b^{n + 1} - a^{n + 1} > (n + 1) a^{n} (b - a)
\end{align*}

(2)

我们有
\begin{align*}
  a_{n}     & = \left(1 + \frac{1}{n}\right)^{n + 1}    \\
  a_{n + 1} & = \left(1 + \frac{1}{n +1}\right)^{n + 2}
\end{align*}
我们需要证明
\begin{align*}
  a_n \geq a_{n + 1}
\end{align*}
令
\begin{align*}
  a & = 1 + \frac{1}{n +1} \\
  b & = 1 + \frac{1}{n}
\end{align*}
利用不等式,我们有
\begin{align*}
  \left(1 + \frac{1}{n}\right)^{n + 1}
   & > (n+1)\left(1 + \frac{1}{n +1}\right)^n \left(1 + \frac{1}{n} - 1 + \frac{1}{n +1}\right)                                     \\
   & = (n+1)\left(1 + \frac{1}{n +1}\right)^n \left(\frac{2n + 1}{n(n +1)}\right)                                                   \\
   & = \left(1 + \frac{1}{n +1}\right)^{n + 2} \frac{(n + 1)\left(\frac{2n + 1}{n(n +1)}\right)}{\left(1 + \frac{1}{n +1}\right)^2} \\
   & = \left(1 + \frac{1}{n +1}\right)^{n + 2}  \frac{2n^3 + 5n^2 + 4n + 1}{n^3 + 4n^2 + 4n}                                        \\
   & > \left(1 + \frac{1}{n +1}\right)^{n + 2}
\end{align*}
于是$a_n > a_{n + 1}$,数列$\{a_n\}$单调递减。

\section*{10}

由习题9可知
\begin{align*}
  (1 + \frac{1}{n})^{n + 1}
\end{align*}
单调递减且有界。
又因为
\begin{align*}
  \lim\limits_{n \to \infty} (1 + \frac{1}{n})^{n + 1} = e
\end{align*}
于是可得
\begin{align*}
  \lim\limits_{n \to \infty} (1 + \frac{1}{n})^{n + 1} = \inf{(1 + \frac{1}{n})^{n + 1}} = e
\end{align*}
由下确界的定义可知,任意$n \in \mathbb{N}^+$,都有
\begin{align*}
  e < (1 + \frac{1}{n})^{n + 1}
\end{align*}

我们有
\begin{align*}
  (1 + \frac{1}{n})^{n + 1} & = (1 + \frac{1}{n})^{n} (1 + \frac{1}{n})                        \\
                            & = (1 + \frac{1}{n})^{n} + (1 + \frac{1}{n})^{n}\cdot \frac{1}{n}
\end{align*}
因为$\lim\limits_{n \to \infty} (1 + \frac{1}{n})^{n} = e = \sup\{(1 + \frac{1}{n})^{n}\}$,
于是可得,对任意$n \in \mathbb{N}^+$都有
\begin{align*}
  (1 + \frac{1}{n})^n < e
\end{align*}
所以
\begin{align*}
  (1 + \frac{1}{n})^{n + 1} & = (1 + \frac{1}{n})^{n} (1 + \frac{1}{n})                        \\
                            & = (1 + \frac{1}{n})^{n} + (1 + \frac{1}{n})^{n}\cdot \frac{1}{n} \\
                            & < (1 + \frac{1}{n})^{n} + \frac{e}{n}                            \\
                            & < (1 + \frac{1}{n})^{n} + \frac{3}{n}
\end{align*}
综上可得
\begin{align*}
  e \leq (1 + \frac{1}{n})^{n + 1} <  (1 + \frac{1}{n})^{n} + \frac{3}{n}
\end{align*}
可得
\begin{align*}
  0 < e - (1 + \frac{1}{n})^n < \frac{3}{n}
\end{align*}
即
\begin{align*}
  \left|e - (1 + \frac{1}{n})^n\right| < \frac{3}{n}
\end{align*}

\section*{11}

(1)先证明$\forall n$有$a_n > b_n$。

对$n$进行归纳。

$n = 1$时,由题设可知$a_1 > b_1 > 0$。\\
归纳假设,$n = k$时,$a_k > b_k > 0$成立。\\
$n = k + 1$时,
我们有
\begin{align*}
  a_{k + 1} = \frac{a_k + b_k}{2} > \sqrt{a_k b_k} = b_{k + 1}
\end{align*}
归纳完成,命题成立。

(2)证明极限存在。

我们有
\begin{align*}
  a_{n + 1} = \frac{a_n + b_n}{2} < \frac{a_n + a_n}{2} = a_n
\end{align*}
于是可得$\{a_n\}$单减。

我们有
\begin{align*}
  b_{n + 1} = \sqrt{a_n b_n} > \sqrt{b_n b_n} = b_n
\end{align*}
于是可得$\{b_n\}$单增。

又$\forall n \in \mathbb{N}^+$有
\begin{align*}
  b_1 \leq b_n < a_n \leq a_1
\end{align*}
可得$\{a_n\}$单调递减且有界,$\{b_n\}$单调递增且有界,
由单调有界定理可知,两个数列极限存在。

(3)极限相等。

设$\lim\limits_{n \to \infty} a_n = a, \lim\limits_{n \to \infty} b_n = b$。
我们有
\begin{align*}
  a_{n + 1} = \frac{a_n + b_n}{2}
\end{align*}
对上式两边取极限得
\begin{align*}
  a & = \frac{a + b}{2} \\
  a & = b
\end{align*}

\end{document}