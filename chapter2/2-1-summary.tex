\documentclass{article}
\usepackage{mathtools} 
\usepackage{fontspec}
\usepackage[UTF8]{ctex}
\usepackage{amsthm}
\usepackage{mdframed}
\usepackage{xcolor}
\usepackage{amssymb}
\usepackage{amsmath}


% 定义新的带灰色背景的说明环境 zremark
\newmdtheoremenv[
  backgroundcolor=gray!10,
  % 边框与背景一致,边框线会消失
  linecolor=gray!10
]{zremark}{说明}

% 通用矩阵命令: \flexmatrix{矩阵名}{元素符号}{行数}{列数}
\newcommand{\flexmatrix}[4]{
  \[
  #1 = \begin{pmatrix}
    #2_{11}     & #2_{12}     & \cdots & #2_{1#4}   \\
    #2_{21}     & #2_{22}     & \cdots & #2_{2#4}   \\
    \vdots      & \vdots      & \ddots & \vdots     \\
    #2_{#31}    & #2_{#32}    & \cdots & #2_{#3#4}
  \end{pmatrix}
  \]
}

% 简化版命令(默认矩阵名为A,元素符号为a): \quickmatrix{行数}{列数}
\newcommand{\quickmatrix}[2]{\flexmatrix{A}{a}{#1}{#2}}

\begin{document}
\title{总结 2.1}
\author{张志聪}
\maketitle

\section*{1}

\begin{zremark}
  核心:

  1.已知极限,按$\epsilon-N$定义证明。
\end{zremark}

题型(1),
最终的目标是通过只包含$\epsilon$变量的方式,表示出$n$的取值范围(即:$N$)。
但往往数列的每项的表达式复杂,无法直接用$\epsilon$表示出$n$的值。
所以,问题主要围绕着化简数列的技巧展开:
\begin{itemize}
  \item 分式,分母缩小。(例3)(\S2 例1)。
  \item $n$次幂(例4),运用$(1+h)^n \geq 1 + nh$
        或使用二项展开$(1+h)^n \geq 1 + nh + \frac{n(n-1)}{2}h^2$(\S2 例2),进行降幂。
  \item $\frac{1}{n}$次幂(例5)。

        先升到1次幂:

        令
        \begin{align*}
          \alpha_n = a^\frac{1}{n} - a
        \end{align*}

        然后
        \begin{align*}
          \alpha_n + a = a^\frac{1}{n} \\
          (\alpha_n + a)^n = a
        \end{align*}

        左侧降幂,运用(1-2)

  \item 累积形式(例6),以累积中的合适项,代替其他项或部分项,进行放大。
  \item 利用已有的极限。$\S2 例3$。
  % \item 数列的项,表示成等差数列或等比数列(或复合形式)的和,先确定$a_n$的表达式,在进行计算。
  %       $\S2 习题4-(1)(3)$。
\end{itemize}


\end{document}