\documentclass{article}
\usepackage{mathtools} 
\usepackage{fontspec}
\usepackage[UTF8]{ctex}
\usepackage{amsthm}
\usepackage{mdframed}
\usepackage{xcolor}
\usepackage{amssymb}
\usepackage{amsmath}

% 定义新的带灰色背景的说明环境 zremark
\newmdtheoremenv[
  backgroundcolor=gray!10,
  % 边框与背景一致,边框线会消失
  linecolor=gray!10
]{zremark}{命题}

\begin{document}

\begin{align*}
    \sum\limits_{i = 1}^n i & = \frac{n(n+1)}{2}
\end{align*}

\begin{align*}
    \sum\limits_{i = 1}^n i^2 & = \frac{n(n+1)(2n+1)}{6}
\end{align*}

\begin{align*}
    \sum\limits_{i = 1}^n i^3 & = \left(\frac{n(n+1)}{2}\right)^2
\end{align*}

\begin{align*}
    (x^n - 1) = (x - 1)(1 + x + \cdots + x^{n-1})
\end{align*}

\begin{zremark}
    $x > 0$,$n \in \mathbb{N}$,那么
    \begin{align*}
        (1 + x)^n \geq 1 + nx
    \end{align*}
\end{zremark}

\textbf{证明:}

对$n$进行归纳即可。

\begin{zremark}
    $x \geq 1$,$n \in \mathbb{N}^+$,那么
    \begin{align*}
        x^{\frac{1}{n}} - 1 \leq \frac{x - 1}{n}
    \end{align*}
\end{zremark}

\textbf{证明:}

$x = 1$,命题易证。

$x > 1$,对$n$进行归纳,
$n = 1$时,命题成立。\\
归纳假设$n = k - 1$时,命题成立,即
\begin{align*}
    x^{\frac{1}{k - 1}} - 1 \leq \frac{x - 1}{k - 1}
\end{align*}
成立。\\
现在证明$n = k$时,需证明
\begin{align*}
    x^{\frac{1}{k}} - 1 \leq \frac{x - 1}{k}
\end{align*}
不等式换个形式
\begin{align*}
    x \leq (1 + \frac{x - 1}{k})^k
\end{align*}
利用命题1,我们有
\begin{align*}
    (1 + \frac{x - 1}{k})^k \geq 1 + k \frac{x - 1}{k} = x
\end{align*}
归纳完成,命题成立。

\begin{zremark}
    等差求和公式推导: 首项为$a_1$,公差为$q$,项数为$n$,
    求等差序列的求和公式,即$S_n$的表达式。
\end{zremark}

\textbf{证明:}

倒序相加法:
\begin{align*}
    S_n & = a_1 + a_2 + \cdots + a_n                    \\
        & = a_1 + (a_1 + q) + \cdots + (a_1 + (n - 1)q)
\end{align*}
倒序写一遍
\begin{align*}
    S_n & = (a_1 + (n - 1)q) + (a_1 + (n - 2)q) + cdots + a_1
\end{align*}
将两式相加:
\begin{align*}
    2S_n & = (2a_1 + (n - 1)q) + (2a_1 + (n - 2)q) + cdots + (2a_1 + q) \\
    2S_n & = n(2a_1 + (n - 1)q)                                         \\
    2S_n & = n(a_1 + a_1 + (n - 1)q)                                    \\
    2S_n & = n(a_1 + a_n)                                               \\
    S_n  & = \frac{n(a_1 + a_n)}{2}
\end{align*}


\begin{zremark}
    等比求和公式推导: 首项为$a_1$,公比为$q$,项数为$n$,
    求等比序列的求和公式,即$S_n$的表达式。
\end{zremark}

\textbf{证明:}
\begin{align*}
    a_1 & = a_1       \\
    a_2 & = a_1 q     \\
    a_3 & = a_2 q     \\
        & \vdots      \\
    a_n & = a_{n-1} q
\end{align*}
于是,我们有
\begin{align*}
    S_n   & = a_1 + a_2 + \cdots + a_n                                    \\
    q S_n & = a_1 q + a_2 q + \cdots + a_n q = a_2 + a_3 + \cdots + a_n q \\
\end{align*}
两式子相减
\begin{align*}
    (1 - q) S_n & = a_1 - a_n q = a_1 - a_1 q^n \\
    S_n         & = \frac{a_1 - a_1 q^n}{1 - q} \\
                & = a_1 \frac{1 - q^n}{1 - q}
\end{align*}

\begin{zremark}
    二项式展开公式: \\
    对正整数n,有
    \begin{align*}
        (x + y)^n = \sum\limits_{k = 0}^n \binom{n}{k} x^{n - k} y^k
    \end{align*}
    其中
    \begin{align*}
        \binom{n}{k} = \frac{n!}{k!(n - k)!}
    \end{align*}
\end{zremark}

\begin{zremark}
    \begin{align*}
        \frac{1}{2} \cdot \frac{3}{4} \cdot \cdots \cdot \frac{2n - 1}{2n} \leq \frac{1}{\sqrt{2n + 1}}
    \end{align*}
\end{zremark}

\begin{zremark}
    \begin{align*}
        n ! < \sum \limits_{p = 1}^n p !
        < (n - 2)(n-2)! + (n - 1)! + n!
        < 2(n - 1)! + n !
    \end{align*}
\end{zremark}

\end{document}