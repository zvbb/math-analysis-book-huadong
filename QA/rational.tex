\documentclass{article}
\usepackage{mathtools} 
\usepackage{fontspec}
\usepackage[UTF8]{ctex}
\usepackage{amsthm}
\usepackage{mdframed}
\usepackage{xcolor}
\usepackage{amssymb}
\usepackage{amsmath}


% 定义新的带灰色背景的说明环境 zremark
\newmdtheoremenv[
  backgroundcolor=gray!10,
  % 边框与背景一致,边框线会消失
  linecolor=gray!10
]{zremark}{说明}

% 通用矩阵命令: \flexmatrix{矩阵名}{元素符号}{行数}{列数}
\newcommand{\flexmatrix}[4]{
  \[
  #1 = \begin{pmatrix}
    #2_{11}     & #2_{12}     & \cdots & #2_{1#4}   \\
    #2_{21}     & #2_{22}     & \cdots & #2_{2#4}   \\
    \vdots      & \vdots      & \ddots & \vdots     \\
    #2_{#31}    & #2_{#32}    & \cdots & #2_{#3#4}
  \end{pmatrix}
  \]
}

% 简化版命令(默认矩阵名为A,元素符号为a): \quickmatrix{行数}{列数}
\newcommand{\quickmatrix}[2]{\flexmatrix{A}{a}{#1}{#2}}

\begin{document}
\title{有理化}
\author{张志聪}
\maketitle

\begin{zremark}
  分类
\end{zremark}
\begin{itemize}
  \item 分母有理化
  \item 分子有理化
\end{itemize}


\begin{zremark}
  常用办法。
\end{zremark}

\begin{itemize}
  \item (1) 平方差公式推广;

  \begin{align*}
    a - b = (\sqrt{a} - \sqrt{b})(\sqrt{a} + \sqrt{b})
  \end{align*}
  
  \item (2) n次方公式推广;

  \begin{align*}
    a - b
    & = (\sqrt[n]{a} - \sqrt[n]{b})
    [(\sqrt[n]{a})^{n - 1} + (\sqrt[n]{a})^{n - 2}(\sqrt[n]{b})+ \cdots + (\sqrt[n]{b})^{n - 1}]
  \end{align*}
\end{itemize}


\end{document}